\selectlanguage{english}
\begin{abstract}
Le reti aziendali rappresentano un pilastro fondamentale per il funzionamento della società moderna, fondamentali anche per il funzionamento di servizi critici in ambiti sanitari, finanziari e per gli operatori di servizi essenziali. Tuttavia, la loro complessità e la crescente esposizione a minacce informatiche richiedono una progettazione e una gestione rigorosa, basata su principi di sicurezza avanzati e standard consolidati. Questa tesi analizza lo stato dell'arte delle tecnologie e strumenti esistenti nelle reti aziendali moderne, per sopprimere vulnerabilità e identificando errori comuni nelle configurazioni e negligenze nella manutenzione. Verranno anche analizzati i framework internazionali come ISO/IEC 27001, NIST CSF e COBIT2019, il tutto supportato dalla letteratura scientifica. Sono stati esaminati casi studio emblematici, come l'attacco ransomware alla Colonial Pipeline del 2021.

Il lavoro analizza le tecniche odierne per mettere in sicurezza una rete, analizzando principi quali la difesa in profondità, il minimo privilegio e l'isolamento e tecnologie come Next-Generation Firewall, i sistemi di rilevamento collaborativo e la segmentazione avanzata tramite VRF. Vengono approfonditi strumenti di monitoraggio proattivo come SIEM e SOC sono anche mostrate metodologie di manutenzione preventiva, tra cui patch management automatizzato e politiche di controllo degli accessi. Particolare attenzione nella fine dell'elaborato è dedicata all'integrazione di intelligenza artificiale e machine learning per l'analisi predittiva delle minacce e l'ottimizzazione dei processi di risposta.
Il progetto è stato scritto in italiano, ma data la forte presenza della lingua inglese in materia sono presenti inglesismi o parole inglesi.
\end{abstract}

\let\cleardoublepage\clearpage