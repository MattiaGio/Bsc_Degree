\hypersetup{linkcolor=blue, citecolor=red}

\chapter{Conclusioni e Nuove Tecnologie per la Sicurezza delle Reti Aziendali} \label{chap:5}
    In questo capitolo andremo ad analizzare le conclusioni della nostra tesi, descrivendo le nuove tecnologie e le limitazioni e approfondendo anche il caso reale citato inizialmente.
    L'intelligenza artificiale, AI, e il machine learning, ML, stanno rivoluzionando la sicurezza nelle aziende, potenziando la capacità di individuare e gestire le minacce. Grazie a queste tecnologie, molti processi di sicurezza vengono automatizzati, riducendo la necessità di intervento umano e accelerando la risposta a potenziali attacchi. I sistemi di AI analizzano automaticamente i comportamenti sospetti e prevengono violazioni, bloccando gli attacchi in tempo reale. Questo approccio è diventato fondamentale per contrastare minacce informatiche sempre più avanzate e complesse. Un altro modello che sta prendendo sempre più piede è quello dei framework Zero-Trust: l'approccio di questo principio si basa sul “never trust, always verify” ovvero si vuole far capire che tutti gli utenti, esterni e interni all'organizzazione, devono autenticarsi ed essere sempre validati\footnote{Fonte: (\url{https://cybersecurityvalidation.com/what-is-security-validation/})} prima di ricevere i permessi su risorse critiche. 
    Stare al passo in un ambiente dove malware, ransomware e vari tipi di attacchi sono in costante sviluppo non è facile per le organizzazioni, abbiamo visto in precedenza come l'aumento di attacchi di questo tipo ha creato il bisogno per un avanzamento tecnologico per le misure di sicurezza nelle reti. Non è quindi facile adeguarsi, soprattutto a causa dei continui investimenti richiesti e la costante ricerca di personale con le necessarie competenze. Questi fattori rappresentano un ostacolo per la maggior parte delle aziende. 

    \newpage
    
    \section{Machine Learning e Artificial Intelligence}
        Questi strumenti possono essere utilizzati per analizzare il traffico e per l'identificazione di pattern pericolosi. Un esempio di implementazione di intelligenza artificiale nella sicurezza delle reti è quello di Darktrace con il suo prodotto che lo utilizza per rilevare e rispondere alle minacce in tempo reale\footnote{Fonte: Darktrace (\url{https://darktrace.com/products})}. Questi strumenti trovano grande applicazione nell'analisi dei dati. Al giorno d'oggi i sistemi di IDS e IPS, di cui abbiamo parlato in precedenza, sono il collo di bottiglia per molte reti; infatti, questi sistemi utilizzano macchine per analizzare volumi enormi di traffico \cite{ml_intro}. Con sistemi come ML e il Deep Learning, DL, si possono creare algoritmi e strumenti in grado di prevedere il traffico. Questo comporta, non solo un aumento dell'efficienza delle prestazioni della rete, come con gli Access Point nel compiere controlli come degli accessi o il load balancing, ma aiuta a mettere in sicurezza i sistemi e ad alleggerire il carico su macchine e reparti IT. L'AI e reti neurali leggiamo da \cite{ann_paper} come le Artificial Neural Network, ANN, siano state spesso proposte anche loro per la predizione del traffico, dato che teoricamente cattura ogni relazione tra l'output e l'input, non è però di facile implementazione. Il ML, unito ad AI, è capace di scoprire autonomamente e quindi di imparare. Questo è cruciale per la rilevazione di nuovi tipi di attacchi, ancora sconosciuti, grazie ai data set enormi da cui apprende ed è capace di integrarsi anche con banche dati come il MITRE ATT\&CK menzionato precedentemente. Gli IDS basati su questa tecnologia raggiungono livelli di rilevamento soddisfacenti quando sono disponibili dati di addestramento sufficienti e i modelli mostrano una buona generalizzazione, permettendo di identificare sia varianti di attacchi esistenti che minacce completamente nuove, ma anche di ridurre sensibilmente i falsi negativi, alleggerendo quindi il carico. Un vantaggio aggiuntivo è che questi sistemi non dipendono eccessivamente da conoscenze specifiche del dominio, rendendoli più semplici da progettare e implementare. Il deep learning, una branca avanzata del machine learning, si distingue per le sue prestazioni eccezionali, superando le tecniche tradizionali nella gestione dei dati. I metodi di deep learning si differenziano proprio perché apprendono autonomamente le rappresentazioni delle caratteristiche direttamente dai dati grezzi, operando in modalità end-to-end e garantendo un approccio pratico ed efficiente \cite{ml_ids_paper}.
    
        Un altro strumento che beneficia particolarmente di queste tecnologie sono i SIEM, i quali devono monitorare una grande quantità di dati, limitando le prestazioni e richiedendo molte risorse. L'integrazione di intelligenza artificiale e machine learning negli strumenti SIEM sta rivoluzionando il modo in cui le organizzazioni rilevano, analizzano e rispondono agli incidenti di sicurezza. Questo studio \cite{ml_in_siem} esplora il futuro dei SIEM nel contesto di un panorama cybersecurity in continua evoluzione, approfondendo come le aziende possano prepararsi all'adozione di sistemi SIEM potenziati dal ML. Queste soluzioni avanzate amplificano le capacità dei SIEM tradizionali, consentendo di identificare e gestire sia minacce note che emergenti con maggiore efficacia. Per sfruttare appieno il potenziale di queste tecnologie, è essenziale che le organizzazioni sviluppino una strategia dati solida, investano nella formazione di personale qualificato e adottino i SIEM abilitati al ML in modo graduale. Restare aggiornati sulle ultime tendenze in ambito ML e cybersecurity è altrettanto cruciale per massimizzare i vantaggi offerti da questi strumenti innovativi. Come detto nei capitoli precedenti troviamo i SIEM al centro dei SoC, infatti anche queste infrastrutture hanno iniziato a sfruttare il ML per rilevare con più precisione ed efficienza le minacce. Analizzando la letteratura scientifica vediamo come in questi studi \cite{visualization_cyber_incident_paper}, \cite{improving_soc} si parlava già di “SoC Intelligente”, evidenziando come l'analisi predittiva e l'adattamento dinamico alle minacce possano ottimizzare le operazioni nei centri.
        Vediamo anche come in questi lavori \cite{modern_soc}, \cite{aut_in_soc}, \cite{MITRE_and_ml}, viene approfondito il ruolo dell'automazione nei Security Operation Centers, definendola come un pilastro per la cyber security moderna. Queste ricerche dimostrano come l'automazione possa snellire processi complessi e liberare risorse per attività critiche grazie a meccanismi già in uso. Questi modelli hanno ispirato lo sviluppo di framework interni per l'automazione nei SOC, migliorando efficienza e tempi di risposta.
        Un altro aspetto cruciale emerso riguarda la sinergia tra Threat Intelligence e Threat Hunting in contesti abilitati dal ML. L'integrazione di queste discipline permette non solo di contrastare minacce note, ma anche di individuare pattern anomali e attacchi avanzati, potenziando la capacità proattiva delle organizzazioni. Questo approccio trasforma i SIEM da semplici strumenti di monitoraggio a sistemi dinamici, capaci di anticipare rischi e adattarsi a scenari in evoluzione. Per ottenere questa automazione sono necessari algoritmi di ML, in grado di analizzare dati diversi di ogni tipo, infatti  l'elaborazione del linguaggio naturale, NLP gioca un ruolo cruciale, trasformando i dati non strutturati in modelli normalizzati. Questo sistema monitora dinamicamente le attività, identificando comportamenti sospetti che potrebbero sfuggire a metodi tradizionali, migliorando così la capacità di risposta proattiva alle minacce. Per avere un “SoC Intelligente”, come detto in precedenza, dobbiamo sfruttare anche le ANN,  usate per prevedere le vulnerabilità sfruttabili su una rete.
        Come abbiamo detto in precedenza il Machine Learning, è ottimo per manipolare i dati \cite{ai_and_lm}, e grazie ad algoritmi basati su questa tecnica, ottimizzati in base allo scenario, possono essere sfruttati anche per il penetration testing. I pen tester infatti \cite{ml_corrects_human_error}, \cite{the_role_of_humans}, \cite{bias_humans_and_ml} potrebbero non essere accurati e possono farsi influenzare negativamente dal loro istinto, i sistemi ML invece sono più accurati in quanto non si fanno influenzare dal contesto, riuscendo a isolare meglio attacchi reali e falsi positivi. L'utilizzo quindi da parte dei pen tester di questi strumenti, unito con le loro competenze, rappresenta la scelta migliore perché combina i punti di forza distintivi delle due componenti, superando i limiti intrinseci di ciascuna. Da un lato, il Machine Learning offre la capacità di analizzare enormi volumi di dati in tempi rapidi, identificando schemi complessi, anomalie o correlazioni che sfuggirebbero all'analisi umana, soprattutto in scenari dinamici e ad alto carico di informazioni, il che permette anche di ridurre il così detto “rumore” nei dati. L'integrazione quindi tra le due soluzioni è la scelta migliore, riuscendo ad aumentare l'efficienza, riducendo gli errori e aumentando la precisione e riuscendo ad adattarsi anche agli scenari più complessi, infatti grazie all'utilizzo 
        dell'output del ML per generare ipotesi mirate, testarle in modo dinamico e adattare le strategie di difesa in tempo reale, anche in presenza di minacce sconosciute \cite{importance_visualization_ml}, \cite{situational_awarness_ml}.

    \section{Architetture Zero Trust}
        Abbiamo anche parlato di framework Zero-Trust, ossia un principio che presuppone che nessuna fiducia implicita venga concessa ad asset o account utente basandosi esclusivamente sulla loro posizione fisica o nella rete o sulla proprietà degli asset, che si concentra sulla protezione delle risorse, dove autenticazione e autorizzazione, sia del soggetto che del dispositivo, sono funzioni distinte eseguite prima che una sessione verso una risorsa aziendale venga stabilita. Analizzando questo paper~\cite{zero_trust_nist} del NIST capiamo come questa architettura è nata come risposta ai trend delle reti aziendali, come utenti remoti, politiche BYOD, di cui abbiamo parlato e asset cloud non situati entro i confini di reti di proprietà aziendale. Lo Zero Trust si concentra sulla protezione delle risorse, anziché sui segmenti di rete, poiché la posizione della rete non è più considerata l'elemento principale per la postura di sicurezza della risorsa.

        All'inizio di questo capitolo abbiamo parlato di framework Zero-Trust, un approccio che parte dall'idea che non si debba dare per scontata la sicurezza di dispositivi o account solo perché sono “dentro” la rete aziendale o perché appartengono all'organizzazione. Nello Zero Trust, anche se un utente è collegato alla LAN o usa un dispositivo aziendale, non viene automaticamente considerato affidabile: prima di concedere l’accesso a qualsiasi risorsa, sia le persone che i dispositivi devono superare controlli di identità, tramite il processo di autenticazione e verifiche sui permessi, grazie al processo di autorizzazione, indipendentemente da dove si trovano.
        Questo modello è nato per rispondere a scenari moderni come il lavoro da remoto, l'uso di dispositivi personali, i BYOD di cui abbiamo parlato nei paragrafi precedenti e il cloud, dove dati e servizi spesso risiedono al di fuori dei confini tradizionali della rete aziendale. Invece di concentrarsi su “zone sicure” come le VLAN o i firewall perimetrali, lo Zero Trust protegge direttamente le risorse critiche, come file, applicazioni, server, trattando ogni accesso come potenzialmente rischioso. La logica è semplice: oggi un laptop connesso alla rete interna può essere compromesso tanto quanto un device collegato da casa, quindi la posizione in rete non basta più come garanzia di sicurezza.

        
    \section{Limiti e conseguenze}
        Durante questa analisi abbiamo analizzato come una rete aziendale dovrebbe essere implementata, nel caso generale, capendo che non può esistere un modello che vada bene per ogni situazione ma che bisogna adattare in ogni caso diversi principi in base al proprio use case.
        Implementare lo stato dell'arte nella sicurezza delle reti però non è un compito facile e richiede una grande quantità di competenze. Una sezione che abbiamo affrontato è quella sull'importanza della formazione del personale, sia tecnico che non, permettendo una maggiore esperienza nell'ambito. Fare questo investimento per le aziende significa avere un ritorno in termini di sicurezza sulla propria infrastruttura di rete. Per le piccole-medie imprese questo investimento non è facile in quanto può pesare sulla loro economia, ma deve comunque essere preso in considerazione per il corretto funzionamento del business.
        \newpage
        Per le aziende grandi, prevedere corsi per la formazione del personale tecnico e la sensibilizzazione, in termini di sicurezza informatica, del personale meno specializzato, dovrebbe essere all'ordine del giorno. Quando questa pratica viene trascurata possono succedere disastri, come quello citato in precedenza della Colonial Pipeline~\cite{colonial_pipeline_attack}.
        Nella nostra analisi sono stati citati diverse volte dei software open-source\footnote{Fonte: Wikipedia (\url{https://it.wikipedia.org/wiki/Open_source})} i quali potrebbero aiutare le aziende che hanno meno possibilità di investimento, garantendo un livello di sicurezza elevato.

        Il problema non si limita solo ad una questione di costi e di competenze, ma dobbiamo anche tener conto, come abbiamo ripetuto diverse volte durante l'analisi, che i malware e gli attacchi sono sempre più avanzati e diventano sempre più complessi ogni giorno. Infatti, anche gli strumenti citati in precedenza possono aiutarci a prevenire molti di questi, così da alleggerire il carico di lavoro, ma non garantendo al 100\% la sicurezza di un'infrastruttura di rete, tesi sostenuta anche dal ricercatore Jose Nazario in questo paper~\cite{never_ending_loop_conclusions}.



    \section{Riflessioni Finali}
        L'obiettivo di questa analisi, come menzionato in precedenza, è quello di mettere insieme tutte le “best-practice” necessarie per creare una rete sicura all'interno di ogni azienda, mostrando degli esempi di alcuni software, e facendo capire l'importanza dell'implementare tali strumenti, soprattutto per le aziende operatrici di servizi essenziali, OSE\footnote{Fonte: CyberSecurity360 (\url{https://www.cybersecurity360.it/cybersecurity-nazionale/operatori-di-servizi-essenziali-ose-chi-sono-e-quali-obblighi-di-sicurezza-hanno/})}, che spesso tralasciano misure di sicurezza ritenute “meno importanti”. A mio avviso questa analisi mi ha permesso di indagare più a fondo una materia che mi ha sempre incuriosito e mi ha dato gli strumenti per permettermi di rispondere alle domande iniziali. Esistono quindi più modi, standardizzati, per mettere in sicurezza le reti ed è bene adoperarli. Da questo elaborato possiamo anche capire quali sono gli strumenti necessari per definire una rete “sicura” e abbiamo esplorato anche vari mezzi che possono alleggerire e supportare non solo gli amministratori di rete, ma anche tutto il personale IT, aiutando a ridurre al minimo le negligenze. Inoltre mi ha anche lasciato un senso di responsabilità per questa disciplina scientifica, in quanto nella società di oggi è tutto connesso e non curarsi della sicurezza di queste infrastrutture di rete può creare disagi importanti che possono riversarsi anche sulla nostra società.