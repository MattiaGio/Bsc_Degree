\hypersetup{linkcolor=blue, citecolor=red}

\chapter{Simulazione di una Rete Reale} \label{chap:4}

    \section{Esempio di implementazione di sicurezza nella rete di Poste Italiane}

        Analizzando la rete della società Poste Italiane\footnote{Poste Italiane: (\url{https://www.poste.it/})} notiamo che si configura come un caso emblematico dell'applicazione pratica dei principi di sicurezza illustrati in questa tesi, integrando concetti teorici e strumenti operativi per la protezione delle reti.
        A livello organizzativo, Poste Italiane ha istituito una struttura di governance dedicata alla cybersecurity, affidando la responsabilità complessiva della sicurezza informatica al Chief Information Security Officer (CISO). Questa scelta rispecchia il principio, ampiamente evidenziato nel \textbf{Capitolo \ref{chap:3}} di questo elaborato, secondo cui una chiara definizione di ruoli e responsabilità è fondamentale per garantire una gestione efficace della sicurezza.

        Sul versante tecnologico, l'adozione di infrastrutture e metodologie all'avanguardia da parte di Poste Italiane si integra perfettamente con i modelli di progettazione di reti sicure presentati nel \textbf{Capitolo \ref{chap:2}}. L'azienda ha attivato tre poli distinti: il Security Innovation Lab, il Computer Emergency Response Team (CERT) e il Distretto Tecnologico Cyber Security di Cosenza. Il primo di questi centri operativi è impegnato nella ricerca applicata per lo sviluppo tecnologico, il secondo si occupa invece della risposta immediata agli incidenti e il terzo dello sviluppo di prototipi evolutivi per la protezione dei pagamenti elettronici. L'obiettivo ultimo di questa struttura si configura come un esempio tangibile dell'applicazione dei principi della difesa in profondità e dell'approccio “Zero Trust” per attenuare le vulnerabilità, concentrandosi sulla definizione e implementazione di modelli, metodologie e prototipi innovativi per l'analisi delle minacce cyber e la tutela dei dati personali, in modo da rafforzare le capacità difensive e di risposta di Poste Italiane e assicurare una gestione efficace ed efficiente della privacy all'interno del Gruppo~\cite{poste_cybersec}.

        Il framework di sicurezza adottato da Poste Italiane si fonda su una rigorosa definizione di policy, sull'analisi dei rischi e sulla gestione centralizzata degli incidenti, elementi che richiamano direttamente i modelli normativi e metodologici esposti in questa tesi, quali ISO/IEC 27001 per cui troviamo la certificazione~\cite{poste_iso_27001}. Notiamo anche dall'analisi fatta in questo articolo~\cite{poste_articolo_cyber} come la società abbia preso seriamente la sicurezza delle loro reti.\\
        Dalla nostra analisi siamo anche venuti a conoscenza di uno strumento proattivo utilizzato da Poste per mantenere le loro infrastrutture sicure, ossia Microsoft Defender XDR\footnote{Sito: (\url{https://www.microsoft.com/it-it/security/business/siem-and-xdr/microsoft-defender-xdr})}. Si tratta di una suite di strumenti pensata apposta per la difesa aziendale che fornisce una sicurezza multi-livello, andando a proteggere gli end point, le email e le applicazioni SaaS (cloud).
        Nell'infrastruttura del Gruppo è presente anche un sistema di monitoraggio della rete. Tutte le informazioni che transitano nel network della società, provenienti sia da dispositivi BYOD che da apparecchiature di rete interne, vengono analizzate e inviate a un server centrale. Quest'ultimo, grazie all'applicativo precedentemente menzionato, correla gli eventi utilizzando l'integrazione con il ~MITRE ATT\&CK già affrontato. Tale correlazione permette di individuare le attività malevole su tutti i dispositivi della rete e cerca di risolverle automaticamente, tramite intelligenza artificiale, segnalando a prescindere l'incidente agli analisti. Abbiamo nominato anche la protezione delle email, infatti questo software analizza gli allegati e gli URL nelle mail che, se considerati malevoli, vengono rimossi.
        Altra funzionalità fondamentale per gli analisti sono i log che permettono di investigare gli eventi che occorrono nella rete, facilitando la verifica del corretto funzionamento dell'infrastruttura.
        All'interno di questa suite troviamo anche la possibilità di integrazione con un SIEM, Microsoft Sentinel\footnote{Sito: (\url{https://azure.microsoft.com/it-it/products/microsoft-sentinel})}, che oltre a funzionare come un SIEM, di cui abbiamo discusso il funzionamento, permette anche di effettuare delle query su tutti i dati che sono analizzati dal Defender XDR tramite linguaggio KQL\footnote{Fonte: Microsoft (\url{https://learn.microsoft.com/it-it/kusto/query/?view=microsoft-fabric})}.
        Grazie a tool come questi, il carico sul reparto IT si riduce notevolmente: da un lato si minimizza la superficie d'attacco, dall'altro si accelerano i tempi di risposta agli incidenti e di investigazione.  Nella nostra analisi siamo venuti a conoscenza dell'utilizzo da parte di Poste anche di NGFW, IDS e IPS avanzati, integrati con questo strumento appena descritto di cui però non possiamo ottenere i dettagli.

       